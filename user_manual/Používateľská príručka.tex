\documentclass[12pt,a4paper]{report}
\usepackage[top=2.5cm, bottom=2.5cm, right=2cm, left=3.5cm]{geometry}

\usepackage{cmap}
\usepackage[slovak, english, czech]{babel}
\usepackage[utf8]{inputenc}
\usepackage[IL2]{fontenc}
\usepackage{listings}
\usepackage{xcolor}
\usepackage{graphicx}
\usepackage{subcaption}
\usepackage[figurename=Fig.]{caption}
\graphicspath{ {images/} }
\usepackage{float}

\usepackage{color}
\definecolor{lightgray}{rgb}{0.95, 0.95, 0.95}
\definecolor{darkgray}{rgb}{0.4, 0.4, 0.4}
%\definecolor{purple}{rgb}{0.65, 0.12, 0.82}
\definecolor{editorGray}{rgb}{0.95, 0.95, 0.95}
\definecolor{editorOcher}{rgb}{1, 0.5, 0} % #FF7F00 -> rgb(239, 169, 0)
\definecolor{editorGreen}{rgb}{0, 0.5, 0} % #007C00 -> rgb(0, 124, 0)
\definecolor{orange}{rgb}{1,0.45,0.13}      
\definecolor{olive}{rgb}{0.17,0.59,0.20}
\definecolor{brown}{rgb}{0.69,0.31,0.31}
\definecolor{purple}{rgb}{0.38,0.18,0.81}
\definecolor{lightblue}{rgb}{0.1,0.57,0.7}
\definecolor{lightred}{rgb}{1,0.4,0.5}
\usepackage{upquote}

\usepackage{bussproofs}
%\usepackage{lmodern}

\usepackage{Pgfplots}
\usepackage{amsmath}

\usepackage{amsthm}
\newtheorem{veta}{Veta}[section]
\newtheorem{lema}[veta]{Lema}
\theoremstyle{definition}
\newtheorem{definicia}{Definícia}[chapter]
\theoremstyle{remark}
\newtheorem*{poznamka}{Poznámka}

\definecolor{light-gray}{gray}{0.8}

\usepackage{pdfpages}

\usepackage{setspace}
\onehalfspacing
\usepackage{enumitem}

\newcommand*\lif{\mathbin{\to}}
\usepackage{forest}

\forestset{
  smullyan tableaux/.style={
    for tree={
      math content
    },
    where n children=1{
      !1.before computing xy={l=\baselineskip},
      !1.no edge
    }{},
    closed/.style={
      label=below:$\times$
    },
  },
}


\usepackage[hidelinks]{hyperref}

\usepackage{eso-pic}

\newcommand\tab[1][1cm]{\hspace*{#1}}

% CSS
\lstdefinelanguage{CSS}{
  keywords={color,background-image:,margin,padding,font,weight,display,position,top,left,right,bottom,list,style,border,size,white,space,min,width, transition:, transform:, transition-property, transition-duration, transition-timing-function}, 
  sensitive=true,
  morecomment=[l]{//},
  morecomment=[s]{/*}{*/},
  morestring=[b]',
  morestring=[b]",
  alsoletter={:},
  alsodigit={-}
}

% JavaScript
\lstdefinelanguage{JavaScript}{
  morekeywords={typeof, new, true, false, catch, function, return, null, catch, switch, var, if, in, while, else, case, break},
  morecomment=[s]{/*}{*/},
  morecomment=[l]//,
  morestring=[b]",
  morestring=[b]'
}

\lstdefinelanguage{HTML5}{
  language=html,
  sensitive=true,   
  alsoletter={<>=-},    
  morecomment=[s]{<!-}{-->},
  tag=[s],
  otherkeywords={
  % General
  <, >,
  % Standard tags
    <!DOCTYPE,
  </html, <html, <head, <title, </title, <style, </style, <link, </head, <meta, />,
    % body
    </body, <body,
    % Divs
    </div, <div, </div>, 
    % Paragraphs
    </p, <p, </p>,
    % scripts
    </script, <script,
  % More tags...
  <b, </b, <strong, </strong, <small, </small, <i, </i, <canvas, /canvas>, <svg, <rect, <animateTransform, </rect>, </svg>, <video, <source, <iframe, </iframe>, </video>, <image, </image>, <header, </header, <article, </article, 
  <w, </w, <s, </s
  },
  ndkeywords={
  % General
  =,
  % HTML attributes
  charset=, src=, id=, width=, height=, style=, type=, rel=, href=,
  % SVG attributes
  fill=, attributeName=, begin=, dur=, from=, to=, poster=, controls=, x=, y=, repeatCount=, xlink:href=,
  % properties
  margin:, padding:, background-image:, border:, top:, left:, position:, width:, height:, margin-top:, margin-bottom:, font-size:, line-height:,
    % CSS3 properties
  transform:, -moz-transform:, -webkit-transform:,
  animation:, -webkit-animation:,
  transition:,  transition-duration:, transition-property:, transition-timing-function:,
  }
}

\lstdefinestyle{htmlcssjs} {%
  % General design
%  backgroundcolor=\color{editorGray},
  basicstyle={\fontsize{10}{11}\ttfamily},   
  frame=b,
  % line-numbers
  xleftmargin={0.75cm},
  numbers=left,
  stepnumber=1,
  firstnumber=1,
  numberfirstline=true, 
  % Code design
  identifierstyle=\color{black},
  keywordstyle=\color{blue}\bfseries,
  ndkeywordstyle=\color{editorGreen}\bfseries,
  stringstyle=\color{editorOcher}\ttfamily,
  commentstyle=\color{brown}\ttfamily,
  % Code
  language=HTML5,
  alsolanguage=JavaScript,
  alsodigit={.:;},  
  tabsize=2,
  showtabs=false,
  showspaces=false,
  showstringspaces=false,
  extendedchars=true,
  breaklines=true,
  % German umlauts
  literate=%
  {á}{{\'a}}1 {é}{{\'e}}1 {í}{{\'i}}1 {ó}{{\'o}}1 {ú}{{\'u}}1
  {Á}{{\'A}}1 {É}{{\'E}}1 {Í}{{\'I}}1 {Ó}{{\'O}}1 {Ú}{{\'U}}1
  {à}{{\`a}}1 {è}{{\`e}}1 {ì}{{\`i}}1 {ò}{{\`o}}1 {ù}{{\`u}}1
  {À}{{\`A}}1 {È}{{\'E}}1 {Ì}{{\`I}}1 {Ò}{{\`O}}1 {Ù}{{\`U}}1
  {ä}{{\"a}}1 {ë}{{\"e}}1 {ï}{{\"i}}1 {ö}{{\"o}}1 {ü}{{\"u}}1
  {Ä}{{\"A}}1 {Ë}{{\"E}}1 {Ï}{{\"I}}1 {Ö}{{\"O}}1 {Ü}{{\"U}}1
  {â}{{\^a}}1 {ê}{{\^e}}1 {î}{{\^i}}1 {ô}{{\^o}}1 {û}{{\^u}}1
  {Â}{{\^A}}1 {Ê}{{\^E}}1 {Î}{{\^I}}1 {Ô}{{\^O}}1 {Û}{{\^U}}1
  {œ}{{\oe}}1 {Œ}{{\OE}}1 {æ}{{\ae}}1 {Æ}{{\AE}}1 {ß}{{\ss}}1
  {ű}{{\H{u}}}1 {Ű}{{\H{U}}}1 {ő}{{\H{o}}}1 {Ő}{{\H{O}}}1
  {ç}{{\c c}}1 {Ç}{{\c C}}1 {ø}{{\o}}1 {å}{{\r a}}1 {Å}{{\r A}}1
  {€}{{\euro}}1 {£}{{\pounds}}1 {«}{{\guillemotleft}}1
  {»}{{\guillemotright}}1 {ñ}{{\~n}}1 {Ñ}{{\~N}}1 {¿}{{?`}}1
  {á}{{\'a}}1
         {ľ}{{\v{l}}}1
         {Ľ}{{\v{L}}}1
         {í}{{\'i}}1
         {ĺ}{{\'l}}1
         {Ĺ}{{\'L}}1
         {ŕ}{{\'r}}1
         {Ŕ}{{\'R}}1
         {é}{{\'e}}1
         {ý}{{\'y}}1
         {ú}{{\'u}}1
         {ó}{{\'o}}1
         {ě}{{\v{e}}}1
         {š}{{\v{s}}}1
         {č}{{\v{c}}}1
         {ř}{{\v{r}}}1
         {ž}{{\v{z}}}1
         {ď}{{\v{d}}}1
         {ť}{{\v{t}}}1
         {ň}{{\v{n}}}1                
         {ů}{{\r{u}}}1
         {Á}{{\'A}}1
         {Í}{{\'I}}1
         {É}{{\'E}}1
         {Ý}{{\'Y}}1
         {Ú}{{\'U}}1
         {Ó}{{\'O}}1
         {Ě}{{\v{E}}}1
         {Š}{{\v{S}}}1
         {Č}{{\v{C}}}1
         {Ř}{{\v{R}}}1
         {Ž}{{\v{Z}}}1
         {Ď}{{\v{D}}}1
         {Ť}{{\v{T}}}1
         {Ň}{{\v{N}}}1                
         {Ů}{{\r{U}}}1  
}

\makeatletter
\g@addto@macro\@floatboxreset{\centering}
\makeatother
\begin{document}
\chapter*{Používateľská príručka}
Pre spustenie aplikácie sú potrebné tri JavaScriptové súbory:
\begin{itemize}
\item \verb!parser.js!
\item \verb!earley-oop.js!
\item \verb!tokenizer.js!
\end{itemize}

Aplikácia zároveň pre svoje fungovanie potrebuje gramatiku, slovník a konfiguráciu pre jazyk, ktorý parsuje, tieto súbory by sa mali nachádzať vo svojich osobitných podpriečinkoch v priečinku \verb!res!:
\\
\begin{forest}
  for tree={
    font=\ttfamily,
    grow'=0,
    child anchor=west,
    parent anchor=south,
    anchor=west,
    calign=first,
    edge path={
      \noexpand\path [draw, \forestoption{edge}]
      (!u.south west) +(7.5pt,0) |- node[fill,inner sep=1.25pt] {} (.child anchor)\forestoption{edge label};
    },
    before typesetting nodes={
      if n=1
        {insert before={[,phantom]}}
        {}
    },
    fit=band,
    before computing xy={l=15pt},
  }
[
	[parser.js
	]
	[earley-oop.js
	]
	[tokenizer.js
	]
	[res
		[configs $\to$ {\fontfamily{cmr}\selectfont priečinok s konfiguráciami pre dané jazyky}
			[SK.json]
			[EN.json]
		]
		[grammars $\to$ {\fontfamily{cmr}\selectfont priečinok s gramatikami pre dané jazyky}
			[SK.json]
			[EN.json]
		]
		[terminalRules $\to$ {\fontfamily{cmr}\selectfont priečinok so slovníkmi pre dané jazyky}
			[SK.json]
			[EN.json]
		]
	]
]
\end{forest}

Na vytvorenie sparsovaného XML z nejakého textu, je treba v súbore \verb!parser.js! vytvoriť inštanciu triedy \verb!Parser!, ktorá dostáva dva parametre. Prvým z nich je text, ktorý sa bude parsovať a druhým je jazyk vo forme reťazca napríklad "SK" alebo "EN". Ďalej sa na túto inštanciu \verb!Parseru! zavolajú funkcie \verb!buildXML()! a \verb!stringifyTree()! prvá z nich doplní do stromu všetky tagy a druhá vráti hotové XML.

\begin{lstlisting}[style=htmlcssjs]
var p = new Parser("- Tak čo je nové? - spytuje sa zvedavo mama. - Aj sama to vieš, - odpovedá dcéra.", "SK");
p.buildXML();
p.stringifyTree()
\end{lstlisting}

Trieda \verb!Parser! umožňuje aj vytvorenie HTML a JSON reprezentácie stromu pomocou funkcií \verb!getHTMLTree()! a \verb!getJSONTree()!, ktoré ich vrátia ako string, prípadne pomocou funkcií \verb!saveHTMLTree(<path>)! a \verb!saveJSONTree(<path>)!, ktoré ich priamo uložia do súboru.

V prípade, že by bolo potrebné upraviť gramatiku - napríklad pridať pravidlo, tak stačí v priečinku \verb!grammars! v niektorom súbore s gramatikou vytvoriť nový atribút JSON-u vo forme:
 \begin{lstlisting}[style=htmlcssjs]
{
	...,
	"LavaStranaPravidla": "PrvaMoznost | DruhaMoznost | ...",
	...,
}
\end{lstlisting}
Alternatívy, ktoré môžu vzniknúť sa oddeľujú pomocou \verb!|! (zvislá čiara).

V gramatike je možné(a aj nutné) použiť tokeny z prislúchajúceho slovníku v priečinku \verb!terminalRules!, ktoré sú vo forme:
\begin{lstlisting}[style=htmlcssjs]
{
	...,
	"RegexKtoryMatchujeToken": "TypTokenu",
	"^[0-9]+$":"Num",
	...,
}
\end{lstlisting}

Úpravy gramatiky sa väčšinou nepodaria na prvý pokus a preto je vhodné využívať parserom vygenerovaný HTML strom na debuggovanie, prípadne si vypísať stromy alebo hotové XML do konzoly:
\begin{lstlisting}[style=htmlcssjs]
var p = new Parser("Toto je vstupný text", "SK");
p.buildXML();
p.saveHTMLTree("../app/res/trees/tree.html"); // subor otvorit vo webovom prehliadaci
p.printTrees(); // vypise stromy do konzoly
console.log(p.stringifyTree()); // vypise XML do konzoly
\end{lstlisting}
\end{document}